\setcounter{equation}{0}

\clearpage


\chapter{Definition <falls erwünscht ...>}

In dieser Diplomarbeit werden nur affine (meist SISO-) Systeme betrachtet.
Für weiterreichende Untersuchungen der übergeordneteren und allgemeineren
nichtlinearen Systemklasse 

\begin{equation}
\left.\begin{array}{ccl}
\dot{x}(t) & = & \mathbf{f}(x(t),u(t))\\
y(t) & = & h(x(t))\end{array}\right\} \label{eq:SISO-allgemeine-nichlineare-Systembeschreibung}\end{equation}
wird auf \cite{Paper_Minimum-Phase-Nonlinear-Systems-A-New-Definition,Paper_Output-Input-Stability-and-Minimum-Phase-Nonlinear-Systems,Paper_A-New-Definition-of-the-Minimum-Phase-Property-for-Nonlinear-Systems-with-an-Application-to-Adaptive-Control,Paper_Smooth-Stabilzation-Implies-Coprime-Factorization}
verwiesen. Für die Betrachtungen und Überlegungen in dieser Diplomarbeit
sind die folgenden Definitionen für affine Systeme ausreichend. 


\section{Definition des Relativgrades}


\subsection{Relativgrad nichtlinearer SISO-Systeme}

Viele nichtlineare Strecken ...

\newpage

\mbox{}

\thispagestyle{empty}
