\setcounter{equation}{0}


\chapter*{Thesis Outline}

\addcontentsline{toc}{chapter}{\numberline{}{Thesis Outline}} Zunächst
soll ein kurzer Überblick über den Aufbau der Arbeit gegeben werden,
um es dem Leser zu erleichtern, sich zurechtzufinden und besonders
als interessant empfundene Themen direkt nachschlagen zu können. <Eine
Art ABSTRACT der Arbeit, um dem Leser einen Überblick über den Aufbau
zu ermöglichen>

Im ersten Kapitel ... 

Das zweite Kapitel fasst ...

Das folgende dritte Kapitel ...

Das vierte Kapitel ...

Im fünften Kapitel ... 

Das sechste Kapitel ... 

Der Anhang umfasst ...

\clearpage\pagenumbering{arabic} \setcounter{page}{1}


\chapter{Einführung in {}``Adaptive Regelungen'' <beispielsweise ...>}

In der modernen Regelungstechnik stößt man bei komplexer werdenden
Systemen für die keine genauen Modelle erstellt werden können, die
Modellierung zu aufwendig ist oder sich die Systemeigenschaften während
des Betriebes ändern, immer häufiger auf den Begriff {}``adaptive
Regelung oder adaptiver Regler''. Aus dem allgemeinen Sprachgebrauch
ist bekannt - {}``adaptieren'' heißt {}``anpassen'' - somit lässt
sich intuitiv eine Bedeutung herleiten, dass es sich hier, um eine
{}``anpassende Regelung oder anpassenden Regler'' handeln muss. 

Trotzdem erfolgt nun ein Definitionsvorschlag, auf den sich diese
gesamte Arbeit in Bezug auf das Wort {}``adaptiv'' beziehen wird
\cite{Book_Adaptive-Control}:

\begin{quotation}
\textit{{}``Ein adaptiver Regler ist ein Regler mit verstellbaren
Parametern und einem Mechanismus, um diese verstellbaren Parameter
anzupassen. Dieser Regler ist im allgemeinen nichtlinear. Eine Regelung,
die solch einen Regler integriert, wird als adaptive Regelung bezeichnet.''}
\end{quotation}
Mit diesem Wissen kann sich der Leser nun weiter in dieses höchst
interessante und komplexe Thema einlesen. 


\section{Geschichtliche Entwicklung }

Anfang der '50er Jahre stieß man bei der Entwicklung von Autopiloten
in Kampfjets auf Probleme in der Regelung. Aufgrund der vielen unterschiedlichen
Betriebszustände solch eines Flugzeuges mit seinen hohen Geschwindigkeiten,
seinen extremen Flughöhen konnten die Regelungsaufgaben mit einfachen
konstanten Regelungen nicht mehr bewältigt werden. 

...


\section{Motivation und {}``State-of-the-Art''}

Ein Hauptgrund zur Einführung und Entwicklung adaptiver Regelungsstrategien
ist sicherlich, dass Regler benötigt werden, die sich an Änderungen
im System \textit{während} des ... 

\cleardoublepage
